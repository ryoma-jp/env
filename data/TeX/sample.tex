\documentclass[11pt,a4paper]{jsarticle}
%
\usepackage{amsmath,amssymb}
\usepackage{bm}
\usepackage{graphicx}
\usepackage{ascmac}
\usepackage{listings}
\usepackage{plistings}
\usepackage{url}
%
\setlength{\textwidth}{\fullwidth}
\setlength{\textheight}{40\baselineskip}
\addtolength{\textheight}{\topskip}
\setlength{\voffset}{-0.2in}
\setlength{\topmargin}{0pt}
\setlength{\headheight}{0pt}
\setlength{\headsep}{0pt}
%
\newcommand{\divergence}{\mathrm{div}\,}  %ダイバージェンス
\newcommand{\grad}{\mathrm{grad}\,}  %グラディエント
\newcommand{\rot}{\mathrm{rot}\,}  %ローテーション
%
\lstset{
	%プログラム言語(複数の言語に対応,C,C++も可)
 	language = Python,
 	%背景色と透過度
 	%backgroundcolor={\color[gray]{.90}},
 	%枠外に行った時の自動改行
 	breaklines = true,
 	%自動開業後のインデント量(デフォルトでは20[pt])	
 	breakindent = 10pt,
 	%標準の書体
 	basicstyle = \ttfamily\scriptsize,
 	%basicstyle = {\small}
 	%コメントの書体
 	commentstyle = {\itshape \color[cmyk]{1,0.4,1,0}},
 	%関数名等の色の設定
 	classoffset = 0,
 	%キーワード(int, ifなど)の書体
 	keywordstyle = {\bfseries \color[cmyk]{0,1,0,0}},
 	%""で囲まれたなどの"文字"の書体
 	stringstyle = {\ttfamily \color[rgb]{0,0,1}},
 	%枠 "t"は上に線を記載, "T"は上に二重線を記載
	%他オプション:leftline,topline,bottomline,lines,single,shadowbox
 	frame = TBrl,
 	%frameまでの間隔(行番号とプログラムの間)
 	framesep = 5pt,
 	%行番号の位置
 	numbers = left,
	%行番号の間隔
 	stepnumber = 1,
	%右マージン
 	%xrightmargin=0zw,
 	%左マージン
	%xleftmargin=3zw,
	%行番号の書体
 	numberstyle = \tiny,
	%タブの大きさ
 	tabsize = 4,
 	%キャプションの場所("tb"ならば上下両方に記載)
 	captionpos = t
}
%
\title{TeX書き方のサンプル}
\author{Ryoichi Matsumoto}
\date{\today}
\begin{document}
\maketitle
%
%
\section{はじめに}
Texの書き方のサンプルをまとめる

\section{記号}

\subsection{ギリシャ文字}

\begin{table} [ h ]
\caption{大文字}
	\begin{center}
	\begin{tabular}{| l | l || l | l || l | l || l | l |} \hline
		文字 & コマンド & 文字 & コマンド & 文字 & コマンド & 文字 & コマンド \\ \hline
		$A$ & A & $B$ & B & $\Gamma$ & \verb|\|Gamma & $\Delta$ & \verb|\|Delta \\ \hline
		$E$ & E & $Z$ & Z & $H$ & H & $\Theta$ & \verb|\|Theta \\ \hline
		$I$ & I & $K$ & K & $\Lambda$ & \verb|\|Lambda & $M$ & M \\ \hline
		$N$ & N & $\Xi$ & \verb|\|Xi & $O$ & O & $\Pi$ & \verb|\|Pi \\ \hline
		$P$ & P & $\Sigma$ & \verb|\|Sigma & $T$ & T & $\Upsilon$ & \verb|\|Upsilon \\ \hline
		$\Phi$ & \verb|\|Phi & $X$ & X & $\Psi$ & \verb|\|Psi & $\Omega$ & \verb|\|Omega \\ \hline
	\end{tabular}
	\end{center}
\end{table}

\begin{table} [ h ]
\caption{小文字}
	\begin{center}
	\begin{tabular}{| l | l || l | l || l | l || l | l |} \hline
		文字 & コマンド & 文字 & コマンド & 文字 & コマンド & 文字 & コマンド \\ \hline
		$\alpha$ & \verb|\|alpha & $\beta$ & \verb|\|beta & $\gamma$ & \verb|\|gamma & $\delta$ & \verb|\|delta \\ \hline
		$\epsilon$ & \verb|\|epsilon & $\zeta$ & \verb|\|zeta & $\eta$ & \verb|\|eta & $\theta$ & \verb|\|theta \\ \hline
		$\iota$ & \verb|\|itoa & $\kappa$ & \verb|\|kappa & $\lambda$ & \verb|\|lambda & $\mu$ & \verb|\|mu \\ \hline
		$\nu$ & \verb|\|nu & $\xi$ & \verb|\|xi & $o$ & o & $\pi$ & \verb|\|pi \\ \hline
		$\rho$ & \verb|\|rho & $\sigma$ & \verb|\|sigma & $\tau$ & \verb|\|tau & $\upsilon$ & \verb|\|upsilon \\ \hline
		$\phi$ & \verb|\|phi & $\xi$ & \verb|\|xi & $\psi$ & \verb|\|psi & $\omega$ & \verb|\|omega \\ \hline
	\end{tabular}
	\end{center}
\end{table}

\begin{table} [ h ]
\caption{変体文字}
	\begin{center}
	\begin{tabular}{| l | l || l | l || l | l || l | l |} \hline
		文字 & コマンド & 文字 & コマンド & 文字 & コマンド & 文字 & コマンド \\ \hline
		$\varepsilon$ & \verb|\|varepsilon & $\vartheta$ & \verb|\|vartheta & $\varpi$ & \verb|\|varpi & $\varrho$ & \verb|\|varrho \\ \hline
		$\varsigma$ & \verb|\|varsigma & $\varphi$ & \verb|\|varphi &  &  &  & \\ \hline
	\end{tabular}
	\end{center}
\end{table}


\section{数式}
...

\section{図表}

\subsection{表}

\begin{lstlisting}
\begin{table} [ h ]
\caption{サンプル}
	\begin{center}
	\begin{tabular}{| l | l | l |} \hline
		a & b & c \\ \hline
		d & e & f \\ \hline
		g & h & i \\ \hline
	\end{tabular}
	\end{center}
\end{table}
\end{lstlisting}

\begin{table} [ h ]
\caption{サンプル}
	\begin{center}
	\begin{tabular}{| l | l | l |} \hline
		a & b & c \\ \hline
		d & e & f \\ \hline
		g & h & i \\ \hline
	\end{tabular}
	\end{center}
\end{table}


\section{レイアウト}
...

\section{参考文献}

\begin{table} [h]
\caption{参考文献}
	\begin{center}
	\begin{tabular} { p{4cm}  p{8cm} } \hline
		タイトル & URL \\ \hline
		Listings & \url{https://mytexpert.osdn.jp/index.php?Listings} \\
		LaTeX入門/各種パッケージの利用 & \url{https://texwiki.texjp.org/?LaTeX入門%2F各種パッケージの利用} \\
		LaTeXコマンド集 & \url{http://www.latex-cmd.com/} \\
		TeXテンプレート & \url{http://hooktail.org/computer/index.php?TeX%A5%C6%A5%F3%A5%D7%A5%EC%A1%BC%A5%C8} \\
		LaTeXで表のセル内改行はtabularx環境を使うと楽 & \url{https://tgnx8810.wordpress.com/2014/11/29/LaTeXで表のセル内改行はtabularx環境を使うと楽/} \\
		\hline
	\end{tabular}
	\end{center}
\end{table}
%
%
\end{document}